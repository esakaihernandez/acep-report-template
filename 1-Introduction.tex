Here is an example introduction. Best practice for longer reports is to write your main text in a separate \textbf{.tex} file and use 

\section{Format snippets}
Here are various blocks of code you might want to use to generate certain formatted features in your document. 

\subsection{Color boxes}
\begin{center}
\fcolorbox{black}{UAFyellow}{
\begin{minipage}{.9\textwidth}
\textbf{Color boxes can be used to add visual emphasis to a section.}

Use them for simplified explanations for complex topics, or for asides that don't fit with the overall narrative of the report. The color is set to UAFyellow, which is defined in acep.cls by the xcolor package. 
\end{minipage}
} % ends fboxcolor
\end{center}


\subsection{Tables}
We don't currently have a standard format for tables in ACEP reports. To save yourself time, use \url{https://www.tablesgenerator.com} to make things that look nice. 

\subsection{Figures}
Overleaf has a helpful guide for all of the different formatting options for figures. \url{https://www.overleaf.com/learn/latex/Inserting_Images}

\begin{figure}
    \centering
    \includegraphics[width=\textwidth]{Figures/ACEP-logo-sm.png}
    \caption{A figure! It will show up in your table of contents.}
    \label{fig:enter-label}
\end{figure}

